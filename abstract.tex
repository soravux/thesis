\chapter*{Abstract}                      % ne pas numéroter
\phantomsection\addcontentsline{toc}{chapter}{Abstract} % inclure dans TdM

\begin{otherlanguage*}{english}

  Understanding images is needed for a plethora of tasks, from compositing to image relighting, including 3D object reconstruction. These tasks allow artists to realize masterpieces or help operators to safely make decisions based on visual stimuli. For many of these tasks, the physical and geometric models that the scientific community has developed give rise to ill-posed problems with several solutions, only one of which is generally reasonable. To resolve these indeterminations, the reasoning about the visual and semantic context of a scene is usually relayed to an artist or an expert who uses his experience to carry out his work. This is because humans are able to reason globally on the scene in order to obtain plausible and appreciable results. Would it be possible to model this experience from visual data and partly or totally automate tasks? This is the topic of this thesis: modeling priors using deep machine learning to solve typically ill-posed problems. More specifically, we will cover three research axes: 1) surface reconstruction using photometric cues, 2) outdoor illumination estimation from a single image and 3) camera calibration estimation from a single image with generic content. These three topics will be addressed from a data-driven perspective. Each of these axes includes in-depth performance analyses and, despite the reputation of opacity of deep machine learning algorithms, we offer studies on the visual cues captured by our methods. 

\end{otherlanguage*}
