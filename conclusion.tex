%!TEX root = main.tex
\chapter*{Conclusion}         % ne pas numéroter
\phantomsection\addcontentsline{toc}{chapter}{Conclusion} % dans TdM


Throughout this thesis, we show that gathering and analyzing large datasets can bring new insights that can both improve classical methods and build advanced machine learning algorithms. Three main axes of research were presented. First, we proposed a performance prediction framework for the problem of tridimensional surface reconstruction through photometric signal under daylight, a problem we call outdoor Photometric Stereo (PS). Using this framework, we demonstrated that partially cloudy days are more suited to perform PS from photometric signal alone. We also showed that clear days typically does not yield enough photometric information to perform a robust reconstruction. We then presented a learning-based method that augments photometric cues with priors to solve the unstability of PS problem during sunny days. Secondly, we proposed a single image learning-based algorithm for outdoor lighting estimation that is robust to various scene content and exhibits state-of-the-art performance. This method enables automatic photorealistic virtual object insertion and relighting, among others. Lastly, we used a similar approach to create a camera calibration method, which estimates automatically the field or view and the position of the horizon within the image. Applications like image retrieval can thus be automated using this technique and used to streamline compositing operations. All proposed methods work on generic scenes and rely heavily on the priors learned directly from the training data. 

For every project presented, we strove to understand the underlying patterns in the data. We proposed a framework for photometric stereo sensitivity analysis, which can predict reconstruction performance from the sky appearance. We also experimented with our surface reconstruction approach under various configurations to better understand the impact of camera calibration error and the number of input images on the reconstruction performance. Furthermore, the proposed camera calibration estimation method was analyzed through guided backpropagation, which allowed us to better understand the visual cues picked up by the learned model. Even though deep learning models are generally considered \emph{black boxes} by the community, we hope our efforts in explaining the behavior of those machine learning methods have inspired others to continue in this direction.

All methods described in this thesis have a direct application in the entertainment field, notably by enabling photorealistic virtual object insertion and relighting automatically. The proposed techniques reduce the time required for a multimedia artist to create or modify works, as testimony from users of Dimension [...]. The light and camera parameters estimation techniques we developed can also provide additional information to sensors, improving the quality of information provided by decision support systems. Similar systems are used [cite urtasun]. Outdoor surface reconstruction would a 

Au point de vue scientifique, mes projets pavent la voie à l’élaboration de nouvelles techniques exploitant l’apprentissage automatique. Dans plusieurs de mes travaux, je suis le premier à coupler certaines disciplines de pointe telle l’estimation d’illumination avec l’apprentissage profond. Les techniques et idées développées lors de ces travaux peuvent être revisitées afin d’être appliquées à d’autres disciplines. Ceci permettra de continuer le développement de cette expertise et d’en assurer sa portée à long terme. 

It is our hope that this thesis has brought some insights and tools to enable the next generation of deep learned priors to computer vision tasks. 
