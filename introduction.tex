\chapter*{Introduction}         % ne pas numéroter
\phantomsection\addcontentsline{toc}{chapter}{Introduction} % inclure dans TdM

% General


we propose to understand and leverage this rich natural illumination.

and leverage this knowledge to improve algorithms locked in the laboratory until now.

% First chapter
Photometric stereo (PS) is a popular, dense shape reconstruction technique that has matured extensively over nearly 40 years~\cite{woodham-opteng-80} to work with complex materials and lighting conditions~\cite{alldrin-cvpr-08,basri-ijcv-07,johnson-cvpr-11,oxholm-eccv-12}.
%Given the excellent PS results obtained in carefully designed laboratory setups,
Simply put, this technique proposes to recover the 3D surface normals of an object observed under varying illumination.
PS is reputed to give exceptional accuracy on surface normal estimation in fully calibrated environments, where the light is controlled.
%It was also successfully coupled with multiview stereo techniques to provide great 3D reconstruction outdoors~\cite{snavely-ijcv-08}.
Recent investigations have turned to the more challenging problem of outdoor PS under uncontrolled, natural illumination. However, it turns out the sun follows a . To solve this issue, recent approaches proposed to capture images over the course of 6 months\cite{ackermann-cvpr-12,abrams-eccv-12}. This time interval provides enough shifting to the sun plane to constrain correctly the PS problem.

instead of trying to be invariant to natural illumination, 

In this thesis, we propose to leverage its richness and subtleties to solve. Our first main contribution is:

\blockquote{\textbf{Single-Day Photometric Stereo} We present a systematic analysis of the expected performance of PS algorithms in outdoor settings on a single day or less, then propose a new machine learning variant to bring this technique.}




In